\documentclass[letter,12pt]{article}
\usepackage[utf8]{inputenc}
\usepackage{times}
\usepackage[letterpaper, margin=1in]{geometry}
\title{Direct detection of the bulk neutral IGM at reionization}
\author{Bahram Mobasher, Darach Watson}
\date{September 2018}

\begin{document}

\maketitle

\section{Scientific Justification}
A significant prediction of our current cosmology is the existence of the
epoch of reionisation, when the intergalactic medium (IGM), neutral since
the emission of the cosmic microwave background, was ionised by luminous
sources in the redshift interval now known to be $z=6.6-10$ (Planck
Collaboration, 2016, A\&A 596, 108).  %\citep{2016A&A...596A.108P}
%The exploration of the sources of reionisation 
%has made great strides over the past years
%with spectroscopic follow-up yielding increasing numbers of positive
%results at $z\gtrsim7$ (e.g.\ Finkelstein et~al.\ 2013, Nature 502, 524;
%Zitrin et~al.\ 2015, ApJL 810, 12; Oesch et~al.\ 2015, ApJL 804, 30; Song
%et~al.\ 2016, ApJ 826, 113; Stark et~al.\ 2017, MNRAS 464, 469),
%%\citep[e.g.][]{2013Natur.502..524F,2016ApJ...819..129O,2016ApJ...826..113S,2015ApJ...810L..12Z,2015ApJ...804L..30O,2017MNRAS.464..469S,2017ApJ...837L..21L}. 
%with all bar two (Watson et~al.\
%2015, Nature 519, 327; Oesch et~al.\ 2016, ApJ 819, 129)
%%\citep{2015Natur.519..327W} 
While reionization epoch spectroscopy has made great strides in recent
years, with increasing numbers of redshift confirmations via Ly$\alpha$ line emission
(e.g.\ Zitrin et~al.\ 2015, ApJL 810, 12; Oesch et~al.\ 2015, ApJL 804, 30; Song
et~al.\ 2016, ApJ 826, 113; Stark et~al.\ 2017, MNRAS 464, 469), the vast
majority of candidate $z>7$ galaxies do not have detectable Ly$\alpha$ lines
(e.g.\ Finkelstein et~al.\ 2013, Nature 502, 524, Watson et~al.\
2015, Nature 519, 327; Oesch et~al.\ 2016, ApJ 819, 129), probably because even at $z\sim7$, the IGM
is significantly neutral.  The spectra of high redshift galaxies, $\gamma$-ray
bursts, and quasars, all show the expected flux suppression due to
the Ly$\alpha$ forest, and at $z\gtrsim 6$, the complete
suppression of flux due to
intervening clouds of H\,{\sc i} gas (the Gunn-Peterson trough). All of this
evidence is compelling, though somewhat indirect. A key result for reionization
would be the direct detection of the red damping wing of the
general neutral IGM in a source during the epoch of reionization.
To date, however, this has not been achieved because of the large
Str\"{o}mgren spheres of quasars and luminous galaxies, and the
faintness of the continuum in sufficiently distant GRBs (Tanvir et~al. 2009, Nature 461, 1254).
%\citep{2009Natur.461.1254T}.  
In addition, even a quasar with high
column density at this redshift is not enough to exclude the option of a local cloud; a
sample would be needed. A GRB at this redshift
would require extremely high SNR spectrum to differentiate between
host galaxy absorbers and the IGM.  A strongly-lensed, intrinsically faint
galaxy with a systemic redshift from ALMA therefore seems our best option to
detect the IGM directly, since it will have neither of these problems.  It
seems, serendipitously, that we have found such a galaxy with a very large
IGM foreground.

Why KECK/MOSFIRE? Most sensitive instrument in the world at this wavelength.
Why is this different from quasars?
1. The emission from the galaxy is extended over a square kpc, rather than a narrow beam, meaning that a small, intervening cloud, always a concern with quasar sightlines, cannot explain these results. Any cloud large enough to explain the high column density and extent would be more massive than all the gas in the host and would be unstable gravitational collapse.
2. The Str\"{o}mgren sphere of the galaxy is small, in contrast to quasars, meaning that the galaxy's close IGM is not substantially altered by it.


\subsection{Immediate Objective}
This very ambitious programme aims to use a
strongly lensed galaxy at $z\sim7$ to directly detect the damped Ly$\alpha$
signature of the IGM during reionization. In 2014, we completed our analysis of ALMA and
VLT/X-shooter data on a $z=7-8$ galaxy candidate, gravitationally lensed by
the galaxy cluster Abell\,1689.  With only very low SNR, we
obtained a coarse optical/NIR spectrum (Fig.~1) that confirmed the high
redshift nature of the galaxy and pinned the redshift to $z=7.5\pm0.2$. 
The break can only be due to Ly$\alpha$ and is not a 4000\,\AA\ break
because of its depth and the blue slope of the continuum.  While the break is clearly visible in Fig.~1, the redshift
uncertainty reflects that we needed to bin the spectrum substantially to
recover the continuum emission.  This galaxy, because of the lensing
magnification of a factor of 9 (Bradley et~al.\ 2008),
%\citep{2008ApJ...678..647B},
is relatively low luminosity, with an intrinsic absolute magnitude of
$M_{\rm UV}\sim-20$.  It is a magnitude below $L*$ at these redshifts. 
Our Cycle~0, 1, and 2 ALMA observations in bands~6 and 7 showed a clear
detection of the source in dust continuum emission with about
$2\times10^7$\,M$_\odot$ of dust and an infrared-measured SFR of
$\sim12$\,M$_\odot$\,yr$^{-1}$ (Watson et~al.\ 2015; Knudsen et~al.\ 2017,
MNRAS 466, 138). Its modest SFR and dustiness make this a good candidate for a galaxy with a small Str\"omgren sphere.
%\citep{2015Natur.519..327W,2017MNRAS.466..138K}.

\hspace{1cm} In the ALMA Cycle~3 data that we are
currently analysing, we find a strong emission line with
significant velocity structure
(Fig.~2). We infer this to be the [C\,{\sc
ii}]\,157.7\,$\mu$m emission line at a redshift $z=7.132$. The line is bright, with a flux of approximately 4\,Jy\,km\,s$^{-1}$.
(The only other possible line ID consistent with the X-shooter break is [O\,\textsc{i}]\,2060.07\,GHz. However, this is very unlikely because the line would be at least an order of magnitude brighter than expected for this galaxy (Cormier et~al.\ 2015, A\&A 587, 53), %\citep{2015A&A...578A..53C}.
and we would expect to detect CO(3--2) easily in our existing data from the Green Bank Telescope (GBT) and we do not (Knudsen et~al. 2017). %\citep{2017MNRAS.466..138K}.
Very high-$J$ CO transitions are excluded. 
%because the atomic lines would be far brighter than ($J\sim19$) CO transitions and they and other CO transitions would be detected in our ALMA data.)
The [C\,\textsc{ii}] identification gives a flux consistent with expectations for this galaxy (De Looze et~al.\ 2011, MNRAS 416, 2712) %\citep{2011MNRAS.416.2712D}
and the expected $L_{\rm CO(3-2)}/L_{\rm [C\,\textsc{ii}]}$ ratio (Stacey et~al.\ 2010, ApJ 724, 957) %\citep{2010ApJ...724..957S}
is also consistent with our GBT spectrum.

\hspace{1cm} A redshift $z=7.13$ is strongly inconsistent with our Ly$\alpha$
break redshift from VLT/X-shooter (Fig.~1), and lies more than
13000\,km/s from the best fit.  A $\log(N_{{\rm H\,I}}/{\rm cm}^2)\simeq23$
at $z=7.13$ would yield an apparent break in very low SNR data at
$z\sim7.5$.  We believe this is the damping wing of the neutral IGM observed
for the first time.  Such a column density is consistent with simulations of
neutral gas densities in a reasonable fraction of lines of sight at this
redshift (Kaurov \& Gnedin 2015, ApJ 810, 154).  The column cannot be
associated with gas from the galaxy itself since the galaxy would be
completely dust obscured ($A_V \sim 60$), given the dust-to-gas ratio for
this galaxy (Watson et~al.\ 2015), furthermore the column density is about
two orders of magnitude larger than damped Ly$\alpha$ absorbers (DLAs)
observed in galaxy spectra to date.  To have such a large column DLA in this
galaxy would require the DLA cloud to be pristine and to be foreground to
essentially the entire galaxy.  This is very unlikely, requiring a
H\,\textsc{i} cloud of at least the same mass as the entire galaxy gas mass
to be gravitationally infalling and in its foreground.  Column densities
approaching this value have been observed in about a percent of $\gamma$-ray
burst (GRB) afterglow DLAs (Tanvir et~al.\ in prep.), however, GRB-DLAs are
pencil beams directly out of young star-forming regions with a covering
fraction of the galaxy light of at most a few percent, and again, those
sightlines are always significantly dust extinguished.  We conclude
that the red shifting of the break away from the systemic redshift is due to
a large neutral IGM component.  This represents a unique opportunity to
detect the damping wing of the neutral IGM during reionization.

\hspace*{1cm} We request a KMOS observation of the galaxy to investigate the
redshift discrepancy.  The observation is designed to be deep enough to get
a direct detection of the IGM damping wing.  We have calculated the expected
spectrum using the KMOS ETC (Fig.~1) and we can clearly discriminate a
damping wing compared to a sharp break and measure the column density in 10
hours exposure.  This wavelength range (990--1030\,nm) is particularly
unfavourable for X-shooter since it lies right in the region of the dichroic
between the VIS and NIR arms. 
% It is worth noting that the JWST will not do
%better than KMOS (see section 8 below). We would have to wait for E-ELT to do this better.
limits.


\section{Technical Remarks (2 pages max)}
\WhyLunarPhase{Very faint target, but NIR observations only, so grey time is acceptable.}

\WhyNights{The required time is calculated based on detecting the  shape of
a Ly$\alpha$ damping wing with a width of $\sim40$\,nm.  We require 10 bins
across the wing, i.e.\ 4\,nm per bin, with a SNR of $\sim3$.  The
KMOS ETC gives for a $J_{\rm AB}=25.0$\,mag source (the HST F125W detection
was $25.0\pm0.13$), extended over 1\,square arcsecond area (Watson et~al. 
2015).  The number of pixels in the reference area is 25.  A SNR=0.54/pixel
is found the integrated source and 0.143\,nm/pixel spectral dispersion in a
21\,hour observation (16 hours exposure time) with 300\,s DIT and NDIT=192,
at a reference wavelength of 1025\,nm.  This corresponds to a SNR over a
4\,nm bin of $\sim2.9$ taking into account that approximately half of the
spectral band is heavily affected by sky emission lines.  Closer to the
middle of the trough, the SNR will drop giving a lower SNR in the wing.  We
used the KMOS ETC with model galaxies, one cutoff at $z=7.13$ (Meiksin 2006)
and one with a $\log(N_{{\rm H\,I}}/{\rm cm}^2)=23.1$ DLA, to represent the
IGM, at $z=7.13$, and determined the signal-to-noise ratio spectrum, which
includes instrument and sky emission and absorption contributions.  

From these SNR spectra we simulated the output spectra and we show these
simulations in Fig.~1 (right). Based on these simulations we can clearly
discriminate between a break and a damping wing at $z=7.13$ with 16 hours on source
exposure.
We can also fit the damping wing to measure the column density. Our DDT
observations were affected by a problem with the requested telescope dither,
removing half of the usable exposure time (just 1.2 hours on source of
usable data) and rendering our sky subtraction strategy worse than expected.
 
To remove the background and optimize the SNR on our high-redshift object we
will adopt the following strategy. We allocate 2 IFUs per
target separated by 20--40 arcsec and use a telescope nodding sequence ABAB
with 300sec exposure each, to move the target from A to B position (similar
to a nodding along the slit). This will optimize the sky
subtraction (i.e. one of the 2 IFUs is always monitoring the sky background)
and maximize the science exposure, since the science target is observed 100\%
of the time. 
%The final stack will be
%obtained after sky subtraction by combining the signal in the 2 IFUs. 
Twenty minutes overhead per OB inferred directly from
P2PP gives a total of 21 hours.
The remaining IFUs will be used to observe photometrically selected lensed galaxies at typically $z=1 -
1.5$. At those redshifts we will target [OII], [OIII]
and $H\beta$. Given the significant exposure time for
many of them we will be able not only to determine redshift but also to
derive meaningful spatially resolved information and dynamical properties.

%Surprisingly, our exposure time calculation with JWST/NIRSpec (confirmed by
%consultation with P.~Jakobsen, NIRSpec insrtrument PI) shows that
%even JWST will not do better than KMOS.  This is because in both cases the
%SNR limitations are the read-out and sky background noise.  For JWST the
%source is extended over a square arcsecond, i.e.\ $\sim100$\,pixels making
%the read-out noise much larger compared to the point source sensitivity. 
%For KMOS, there are fewer pixels, but the sky background is brighter and we
%have to use relatively short exposures to ensure a good sky subtraction.  We
%will therefore have to wait till the E-ELT epoch to make this detection
%unless we can do it here with KMOS.

}

\TelescopeJustification{There are very few instruments in the world capable
of detecting the wing at this redshift.  We require an 8\,m-class telescope
because the target is so faint, and we require an efficient detector at
1000\,nm.  X-shooter has a factor of 2--3 dip in efficiency at precisely
this wavelength range due to using the low sensitivity first and last
spectral orders of the NIR and VIS arms respectively.  The FORS detector has
too low sensitivity at these wavelengths.  However KMOS's $Iz$ grating has
the sensitivity to do this in a reasonable integration time.}

\ModeJustification{N/A. Service mode is preferred due to the good observing
conditions required, however visitor mode is acceptable.}
%\DDTJustification{The science is extremely high impact and is currently a very hot topic as we are beginning to find more and more spectroscopically-confirmed reionization-epoch galaxies, almost all of which have bright Ly$\alpha$ emission lines. This means that none of these $z=7-9$ galaxies have a neutral IGM around them. We have just got the ALMA data for a confirmed $z=7$ galaxy, and it's given us this large redshift discrepancy, pointing to a large neutral IGM. If we detect it here, it will be the first time the damping wing of the neutral IGM will have been detected directly -- a key prediction of the reionization epoch that has so far eluded us. The target has best visibility in Mar/Apr, meaning that if it is not observed soon, we will have to wait a year to observe it in the normal round. We believe the science case is sufficiently remarkable to justify observing immediately.}

\Calibrations{standard}{}

\LastObservationRemark{
VLT: 299.A-5004(A) this proposal as a DDT (PI: Watson)

ALMA: 2015.1.01406.S (PI: Watson): The ALMA data presented in this proposal (Fig.~2). Just received, currently being analysed.

099.A-0292, 098.A-0502, 097.A-0269, 096.A-0496 (PI: Richard) MUSE/GTO observations of lensing clusters. Data has been published in Bina et~al.\ (2016), Patricio et~al.\ (2016), Mahler et~al.\ (2017) and Lagattuta et~al.\ (2017).

ALMA: 2016.1.00329.S (PI: Micha{\l}owski): the data have been reduced and are being analysed.

APEX: 096.D-0280, 096.F-9302, 097.F-9308 (PI: Micha{\l}owski): the
data have been reduced and analysed. The results for one galaxy is
published in Micha{\l}owski et~al.\ (2014, A\&A, 562, 70). The survey paper is being written.
}

\RequestedDataRemark{We have deep X-shooter data for this target, but as explained above, in this specific band $990-1030$\,nm, where the X-shooter dichroic is, only KMOS has the sensitivity to make this detection.
We have 3 hours of KMOS data observed, but with only about half of it usable, as
explained above.}

\RequestedDuplicateRemark{No duplications.}


\Publications{
  Watson D. et~al., 2015, Nature, 519, 327: A dusty, normal galaxy in the epoch of reionization
  \smallskip\\
  Knudsen K.~K. et~al., 2017, MNRAS 466, 138: A merger in the dusty, z = 7.5 galaxy A1689-zD1?
  \smallskip\\
  Micha{\l}owski, M.~J.,  2015, A\&A, 577, 80: Dust production 680-850 million years after the Big Bang
  \smallskip\\
  Mahler, G., Richard, J., Cl\'ement, B., Lagattuta, D., Schmidt, K., et al. 2017, MNRAS submitted, astro-ph/1702.06962: Strong lensing analysis of Abell 2744 with MUSE and Hubble Frontier Fields images
  \smallskip\\
  Lagattuta, D., Richard, J., Cl\'ement, B., Mahler, G., et al. 2017, MNRAS submitted, astro-ph/1611.01513: Lens Modeling  Abell 370: Crowning the Final Frontier Field with MUSE
  \smallskip\\
  Bina, D., Pello, R., Richard, J., et al., 2016, A\&A, 590, 14: MUSE observations of the lensing cluster Abell 1689
  \smallskip\\
  Patr\'\i cio, V., Richard, J., et al., 2016, MNRAS, 456, 4191: A young star-forming galaxy at z = 3.5 with an extended Lyman α halo seen with MUSE
  \smallskip\\
  Richard, J., Patricio, V., Martinez, M. et al., 2015, MNRAS 446, 16: MUSE observations of the lensing cluster SMACSJ2031.8-4036: new constraints on the mass distribution in the cluster core
  \smallskip\\
  Stark, D., Richard, J., et al. 2015, MNRAS, 450, 1846, Spectroscopic detections of CIII]1909 at $z\sim$6-7: A new probe of early star forming galaxies and cosmic reionisation
  \smallskip\\
  Alavi, A., Siana, B., Richard, J. et al., 2014 ApJ 780, 143: Ultra-faint ultraviolet galaxies at $z\sim2$
  behind the lensing cluster A1689: The luminosity function, dust extinction and star formation rate density
  \smallskip\\
  Vanzella E., et~al., incl.\ Christensen, 2017, ApJ submitted: Magnifying the Early Episodes of Star-Formation: A pair of low-metallicity super-star clusters at z=3.2222
  \smallskip\\
  Tilvi V., et~al., incl.\ Christensen, 2016, ApJ, 827, 14: First results from faint infrared grism survey: first simultaneous detection of Lyman-a emission and Lyman break from a galaxy at z = 7:51
}

\Target{A}{A1689-zD1}{13 11 29.93}{-01 19 18.7}{10}{25 ($J_{\rm AB}$)}{$1^{\prime\prime}$}{}{}


\subsection{Targets and Exposures:} 
Specify targets or fields and their coordinates, magnitudes (if known), exposure times, justification for lunar phase, and justification for number of requested nights. 

\subsection{Backup Program:}
Provide a short abstract of backup program(s) for poor observing conditions.

\subsection{Supplementary Observations:}
If supplementary observations from other observatories (including Lick or space) are needed, please describe. Specify how essential are the other data for completion of program. Make clear the role of non-UC collaborators and list any of their time at other observatories that may be part of this project.

\subsection{Status of Previously Approved Keck Programs:}
Review status of previously approved programs (not including the current project), including Title, PI, number of allocated nights, status of data and analysis, and publications to date. For papers in preparation, list an estimated time to publication. Do not go back more than five years. 


\section{Optional, but highly recommended, subsections (2 pages max)}

\subsection{Path to Science from Observations:}
Explain briefly the path from observations to science, including plans for how the data will be taken, reduced, analyzed, and written up for publication. Include a brief description of who will be responsible for which tasks. If theoretical models or other data are needed, specify how these will be acquired. 

\subsection{Technical Concerns:}
When appropriate, specify technical concerns or problems and provide sufficient information on how they will be addressed to enable the TAC to evaluate the feasibility of the project. 

\subsection{Experience and Publications:}
Brief account of the experience of the proposer(s) with instrument and science. Provide no more than a one page list of publications relevant to this proposal that the PI has been involved with. Mark those using Keck or Lick data with an asterisk (*). 

\subsection{Resources and Publication Timescale:}
Please describe what resources (such as computing, personnel, and grant funds) will be applied to this project. What is the expected timescale for publications? Be as specific as possible regarding topics for planned papers.

\subsection{Service to UCO (new since 2017B):}
The UCO Advisory Committee has requested that the following brief section be included: 
Please describe any activities you have carried out in the service of UCO within the last two years e.g. enabling new or upgraded instrumentation, technology development, software development, committee service, fundraising, conference organization, public outreach, etc.

\end{document}
